% !TeX spellcheck = en_GB
\documentclass[a4paper,kul]{kulakarticle} %options: kul or kulak (default)

\usepackage[utf8]{inputenc}
\usepackage[english]{babel}
\usepackage[T1]{fontenc}
\date{Academic year 2023 -- 2024}
\address{
        Bachelor of Engineering Technology \\
        Complex Digital Design (B-KUL-T3WDO2)\\
        Balasch Masoliver Josep}
\title{Lab report Complex Digital Design}
\author{Robbe Decapmaker, Kobe Michiels}
\usepackage{hyperref}
\usepackage{graphicx}
\usepackage{amsmath, amssymb, amsthm}
%\usepackage{siunitx} %cdd_report.tex: error: 18: File `siunitx.sty' not found. \usepackage
\usepackage{flafter}
\usepackage{pdfpages}
\usepackage{pgfplots}
\usepackage{caption}
\usepackage{subcaption}
\usepackage{datetime2}
\usepackage{subfiles}
\usepackage{multicol}
\setlength{\columnsep}{1cm}
\newcommand{\Lapl}{\ensuremath{\mathcal{L}}}

\usepackage{titlesec}
\usepackage[shortlabels]{enumitem}

\setcounter{secnumdepth}{4}

\titleformat{\paragraph}
{\normalfont\normalsize\bfseries}{\theparagraph}{1em}{}
\titlespacing*{\paragraph}
{0pt}{3.25ex plus 1ex minus .2ex}{1.5ex plus .2ex}

\hypersetup{
	pdftitle={Lab report Complex Digital Design},
	pdfsubject={},
	pdfauthor={Robbe Decapmaker, Kobe Michiels},
	pdfkeywords={}
}

\begin{document}

\maketitle
\section{Features}

% A summary of the features of your final project. For the mandatory assignment, you should provide the maximum ADDER_WIDTH tolerated by your design and the number of cycles your design requires to complete a 512-bit addition. If you have completed any of the optional assignments, make sure to summarize them as well in this part.

\subsection{Carry Lookahead Adder (CLA)}

We have a design for a CLA which synthesizes and implements for any amount of bits required by the design. Everything auto generates based of the given adder width. We did not test the limit of this adder implementation, since the size of the adder increased substantially making it a bad choice for increasing our adder sizes when looking from an area saving perspective. There is potential for these adders to be integrated in our VCSA (see section \ref{sec:VCSA}), but due to a lack of time, this was not further investigated. Some experiments we performed however showed potential to reach a 256 bit adder. 

\subsection{Uniform size Carry Select Adder (UCSA)}

As with the CLA, our UCSA synthesizes and implements for any adder width required by a design. We do this by calculating the optimal amount of blocks and block widths at synthesis time. As was detailed in lecture \#3A, we make the block size equal to the square root of the adder width. This ensures the optimal usage of our logic and minimizes idle time. This adder can be set to 64 bit without issue, achieving timing closure in about 7ns. When asking for 128 bit, we can not achieve timing closure within the required 8ns. Unfortunately we get a Worst Negative Slack (WNS) of -0.8ns. 

\subsection{Variable size Carry Select Adder (VCSA)}
\label{sec:VCSA}

The final adder we made, using the process described in section \ref{sec:tech_desc}, is able to achieve a 128 bit addition using a VCSA structure. It achieves this with a WNS of 0.7ns leaving some headroom in our timings. This reduces the required cycles for our mp\_adder to just 6 cycles, providing about a 5 times improvement compared to the reference design. 

\subsection{Status LED}

We used the LEDs of the pynq board to indicate the status of our design. It is able to show the status of the UART transfers of operand A and B (using LED 1 and 2) and whether or not the addition has been made successfully. This allows for easy debugging because it shows clearly which step of the process has gone wrong. 


\newpage
\section{Technical description}
\label{sec:tech_desc}

% A technical description of your main arithmetic designs in the final project. For the improved combinational adder, you should explain the strategy you have followed and provide a high-level diagram of the design. The diagram should be similar to the ones seen in the lectures. For the optional features, you should also provide an explanation and, if required for comprehension, a high-level diagram.

The implementation of our 128 bit variable sized carry select adder follows the principles described in Lecture \#3A. We found however, that making a practical implementation of it was more difficult than expected. This is because there are a lot of variables at play which have an influence on the ability of the circuit to get timing closure. To this end, we wrote a python script (\texttt{variable\_generator.py}) which takes in a few settings (defined in \texttt{settings.yml}). This allowed us to make rapid iterations of our structure by tweaking just a few settings:
\begin{description}
	\item[adder.width] The width of the adder (128 bit)
	\item[adder.start\_block] The width of the first ripple carry adder 
	\item[adder.start\_repetition] How many times the width of the first block should be repeated before expanding the subsequent blocks
	\item[adder.max\_block\_width] How wide should a block maximally be to prevent excessive fan-out. 
	\item[adder.block\_size\_increment] How much should we increase the full adder count per block when increasing the block size
\end{description}
These parameters were chosen because we believe they give us good control over the critical path with respect to logic (amount of multiplexers) and propagation (fan-out) delays. \\
\\
In the end we ended up with a configuration as seen in figure \ref{fig:128bit-adder}. The specific performance is detailed in the following sections. 
\begin{figure}[h]
	\centering
	\includegraphics[width=1\linewidth]{images/128bit-ADDER}
	\caption{Schematic of the variable size carry select adder (N=128)}
	\label{fig:128bit-adder}
\end{figure}

\section{Performance evaluation}

% A performance evaluation of your arithmetic designs, including their worst-case delays (as detailed by Vivado in the post-synthesis report) and the area costs (as detailed by Vivado in the post-synthesis utilization report). 

The performance evaluation of our arithmetic design for an ADDER\_WIDTH of 128 bits consists of an evaluation of the worst-case delay and the area costs. We will firstly generate a post-synthesis Timing Summary to understand if the design met the timing constraints. This Timing Summary is visible on figure \ref{fig:timing-128-bit}. We get a Worst Negative Slack (WNS) of 0.707 ns. Since this is a positive value, we can conclude that there are no timing errors during synthesis.
\\\\
The Intra-Clock Paths in the Timing Report allow us to form a conclusion about the worst case delay. Using the values from figure \ref{fig:paths-128-bit}, we can calculate the worst-case delay as the sum of the logic and net delays of the worst path: 1.749 ns + 5.408 ns = 7.157 ns.
\\
We can also evaluate the number of clock cycles needed for our addition. The latency of this operation is determined by the division of OPERAND\_WIDTH by ADDER\_WIDTH:
\begin{equation}
	OPERAND\_WIDTH/ADDER\_WIDTH = 512/128 = 4 \mathrm{\ clock\ cycles}
\end{equation} 

To evaluate the area costs of our design, we take a look at the post-synthesis utilization report on figure \ref{fig:utilization-128-bit}. We can see that there are 1608 LUTs and 3219 FFs.  

\begin{figure}[h]
	\centering
	\includegraphics[width=0.75\linewidth]{images/timing-128-bit.png}
	\caption{Design Timing Summary for ADDER\_WIDTH = 128}
	\label{fig:timing-128-bit}
\end{figure}

\begin{figure}[h]
	\centering
	\includegraphics[width=0.75\linewidth]{images/paths-128-bit.png}
	\caption{Intra-Clock Paths for ADDER\_WIDTH = 128}
	\label{fig:paths-128-bit}
\end{figure}

\begin{figure}[h]
	\centering
	\includegraphics[width=0.75\linewidth]{images/utilization-128-bit.png}
	\caption{Utilization report for ADDER\_WIDTH = 128}
	\label{fig:utilization-128-bit}
\end{figure}

\section{Comparison }

% A comparison of the obtained performance metrics with respect to the original ones from Lab #3, that is, when MP ADDER uses the ripple_carry_adder_Nb with ADDER_WIDTH = 16. Discuss whether the obtained results (speed improvement and area increases) are in line with the expectations.

The reference design from Lab \#3 with an ADDER\_WIDTH of 16 bits has a Worst Negative Slack of 2.651 ns (figure \ref{fig:timing-16-bit}). This is better then our design which has a WNS of 0.707 ns. The worst-case delay of the reference design is 5.213 ns (figure \ref{fig:paths-16-bit}). Our design has a higher one with a value of 7.157 ns. Our design also has a higher area cost, with 1608 LUTs opposed to 1433 LUTs for the reference design (figure \ref{fig:utilization-16-bit}). However, both designs utilize 3219 FFs. 
\\\\
The higher WNS, worst-case delay and area cost is something we expected. Like we can see on figure \ref{fig:128bit-adder}, our design consists of significantly more adders and multiplexers. This will lead to bigger area utilisation and longer paths. These longer paths are the reason for the higher worst-case delay.
\\\\
The latency of the operation of the reference design is equal to 32 clock cycles. This is 8 times more than the latency of our operation. This is also something we expected because our ADDER\_WIDTH is 8 times higher. This increase in ADDER\_WIDTH results in a decrease in iterations and hence, a decrease in clock cycles.

\begin{figure}[h]
	\centering
	\includegraphics[width=0.75\linewidth]{images/timing-16-bit.png}
	\caption{Design Timing Summary for ADDER\_WIDTH = 16}
	\label{fig:timing-16-bit}
\end{figure}

\begin{figure}[h]
	\centering
	\includegraphics[width=0.75\linewidth]{images/paths-16-bit.png}
	\caption{Intra-Clock Paths for ADDER\_WIDTH = 16}
	\label{fig:paths-16-bit}
\end{figure}

\begin{figure}[h]
	\centering
	\includegraphics[width=0.75\linewidth]{images/utilization-16-bit.png}
	\caption{Utilization report for ADDER\_WIDTH = 16}
	\label{fig:utilization-16-bit}
\end{figure}

\end{document}

