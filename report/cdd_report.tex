% !TeX spellcheck = en_GB
\documentclass[a4paper,kul]{kulakarticle} %options: kul or kulak (default)

\usepackage[utf8]{inputenc}
\usepackage[english]{babel}
\usepackage[T1]{fontenc}
\date{Academic year 2023 -- 2024}
\address{
        Bachelor of Engineering Technology \\
        Complex Digital Design (B-KUL-T3WDO2)\\
        Balasch Masoliver Josep}
\title{Lab report Complex Digital Design}
\author{Robbe Decapmaker, Kobe Michiels}
\usepackage{hyperref}
\usepackage{graphicx}
\usepackage{amsmath, amssymb, amsthm}
%\usepackage{siunitx} %cdd_report.tex: error: 18: File `siunitx.sty' not found. \usepackage
\usepackage{flafter}
\usepackage{pdfpages}
\usepackage{pgfplots}
\usepackage{caption}
\usepackage{subcaption}
\usepackage{datetime2}
\usepackage{subfiles}
\usepackage{multicol}
\setlength{\columnsep}{1cm}
\newcommand{\Lapl}{\ensuremath{\mathcal{L}}}

\usepackage{titlesec}
\usepackage[shortlabels]{enumitem}

\setcounter{secnumdepth}{4}

\titleformat{\paragraph}
{\normalfont\normalsize\bfseries}{\theparagraph}{1em}{}
\titlespacing*{\paragraph}
{0pt}{3.25ex plus 1ex minus .2ex}{1.5ex plus .2ex}

\hypersetup{
	pdftitle={Lab report Complex Digital Design},
	pdfsubject={},
	pdfauthor={Robbe Decapmaker, Kobe Michiels},
	pdfkeywords={}
}

\begin{document}

\maketitle
\section{Introduction}

\section{Features}

% A summary of the features of your final project. For the mandatory assignment, you should provide the maximum ADDER_WIDTH tolerated by your design and the number of cycles your design requires to complete a 512-bit addition. If you have completed any of the optional assignments, make sure to summarize them as well in this part.

\section{Technical description}

% A technical description of your main arithmetic designs in the final project. For the improved combinational adder, you should explain the strategy you have followed and provide a high-level diagram of the design. The diagram should be similar to the ones seen in the lectures. For the optional features, you should also provide an explanation and, if required for comprehension, a high-level diagram.

\section{Performance evaluation}

% A performance evaluation of your arithmetic designs, including their worst-case delays (as detailed by Vivado in the post-synthesis report) and the area costs (as detailed by Vivado in the post-synthesis utilization report). 

\section{Comparison }

% A comparison of the obtained performance metrics with respect to the original ones from Lab #3, that is, when MP ADDER uses the ripple_carry_adder_Nb with ADDER_WIDTH = 16. Discuss whether the obtained results (speed improvement and area increases) are in line with the expectations.

\section{Conclusion}

\end{document}

